% Chapter Template

\chapter{Abelian Gauge Theory} % Main chapter title

\label{Chapter2} % Change X to a consecutive number; for referencing this chapter elsewhere, use \ref{ChapterX}

\lhead{Chapter X. \emph{Abelian Gauge Theory}} % Change X to a consecutive number; this is for the header on each page - perhaps a shortened title

%----------------------------------------------------------------------------------------
%	SECTION 1
%----------------------------------------------------------------------------------------

\section{Local Gauge Invariance}
The previous chapter concluded with a gauge transformation of the Dirac Field that kept it invariant. The gauge transformation was a global one due to the fact that the actual nature of the transformation didn't depend on the position of the particle in space-time. We, however, can't expect all situations that appear in nature to be so. There could be transformations that depend on where the particle is, and the Lagrangian has to remain invariant to that transformation as well. Such transformations are called local transformations and are given by,
\begin{subequations}
\begin{align}
\psi \xrightarrow{} e^{-ieQ{\theta}(x)}\psi\\
A_\mu \xrightarrow{} {A^'}_\mu
\end{align}
\end{subequations}\cite{lahiri04}
The transformed Lagrangian will be \cite{lahiri04}

\begin{equation}
L^'}=\overset{-}{\psi}(i\gamma^\mu\partial_{\mu}-m)\psi-eQ({A^'}_{\mu}-\partial_\mu\theta)\overset{-}\psi\gamma^0 \psi\\
\end{equation}
The Lagrangian will remain invariant if ${A^'}_\mu \xrightarrow{} {A_\mu}+{\partial_\mu}\theta$. This is the familiar gauge transformation of the Electromagnetic field that was introduced in the previous chapter. Thus, upon adding a gauge field term and the Lagrangian of the Electromagnetic Field for logical consistency, we get
\begin{equation}
{L} = \overset{-}{\psi}(i\gamma^{\mu}\partial_\mu-m){\psi}-\frac{1}{4}F_{\mu\nu}F^{\mu\nu}-eQ\overset{-}{\psi}{\gamma}^0{\psi}A_\mu
\end{equation}
This is the Lagrangian of Quantum Electrdynamics or QED.
%----------------------------------------------------------------------------------------
%	SECTION 2
%----------------------------------------------------------------------------------------

\section{Quantum Electrodynamics}
As was shown in the previous section, the Lagrangian of QED is
\begin{equation}
{L} = \overset{_}{\psi}(i\gamma^{\mu}\partial_\mu-m){\psi}-\frac{1}{4}F_{\mu\nu}F^{\mu\nu}-eQ\overset{_}{\psi}{\gamma}^0{\psi}{A_\mu}	
\end{equation}
While $\overset{_}{\psi}(i\gamma^{\mu}\partial_\mu-m){\psi}$ describes the free spin $\frac{1}{2}$ particle and $-\frac{1}{4}F_{\mu\nu}F^{\mu\nu}$ describes the photon field, $-eQ\overset{_}{\psi}{\gamma}^0{\psi}A_\mu$ describes the interaction between the two. An observation to make is that the quantum of electric charge, together with the net total number of charged particles, that enable the interaction as they act as the coupling parameter. This reinforces the fact that only charged particles interact with the Electromagnetic field. The Electromangetic field is also called the bosonic field as they represent photons that act as spin-1 carriers of the field when interacting with an electron current $eQ\overset{_}{\psi}{\gamma}^0{\psi}$.\\
The form of the Lagrangian is Lorentz invariant. Under a Lorentz boost ${\Lambda^mu}_\nu$, the wavefunction of the particle transforms as follows to preserve invariance \cite{halzen84}.
\begin{equation}
S^{-1}{\gamma^\mu}S={\Lambda^\mu}_\nu{\gamma^\nu}
\end{equation}
This equation demonstrated that while the representation of $\gamma^\mu$ changes from frame to frame, the underlying physics remains the same.\\
The solutions to the Hamiltonian of QED that is generated from the Euler-Lagrange equations are Bilinear covariants as they transform according to (2.5). They however form a wide variety of physical quantities as follows \cite{halzen84}
\begin{center}
\begin{tabular}{l|c|r}
\hline
$\overset{_}{\psi}\psi$ & Scalar & Space Inversion:+\\ \hline
$\overset{_}{\psi}{\gamma^\mu}\psi$ & Vector & Space Inversion:-\\ \hline
$\overset{_}{\psi}\sigma^{\mu\nu}$ & Tensor & \\ \hline
$\overset{_}{\psi}\gamma^5\gamma^\mu\psi$ & Axial Vector & Space Inversion:+\\ \hline
$\overset{_}{\psi}\gamma^5\psi$ & Pseudoscalar & Space Inversion:-\\ \hline
\end{tabular}
\end{center}

where $\gamma^5$ is $i\gamma^0\gamma^1\gamma^2\gamma^3$