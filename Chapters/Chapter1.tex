% Chapter 1

\chapter{Free Particle Fields} % Main chapter title

\label{Chapter1} % For referencing the chapter elsewhere, use \ref{Chapter1} 

\lhead{Chapter 1. \emph{Klein-Gordon Field}} % This is for the header on each page - perhaps a shortened title

%----------------------------------------------------------------------------------------

\section{Spin-0 Particles}
Quantum Field Theory is the theory in theoretical physics that is used to study field that are bound by the laws of Quantum Mechanics and Einstein's Theory of Relativity. In Field Theory, a Lagrangian is defined that remains invariant to any coordinate transformation. The Lagrangian thus defined contains all information about the particle that is needed to be known; in classical field theory it contains the past, present, and future forms of the trajectory of the particle and in quantum field theory it contains the nature of evolution of the wavefunction, and thus the probability distribution, of the particle for all time. The Lagrangian for a scalar field (A field with just one quantifier at each point in space-time) is as follows:

\begin{equation}
{L} = \frac{1}{2}({\partial^{\mu}} \Phi)({\partial_{\mu}} \Phi^*) - \frac{1}{2}m\Phi\Phi^*
\end{equation}
This is the Klein-Gordon Field of spin-0 particles. Equation (1.q) is called the Klein-Gordon field after physicists Oskar Klein and Walter Gordon who proposed this model to describe relativistic particles in 1926.\cite{KGE}
The Hamiltonian for such a particle can be derived from the Euler-Lagrange equations of motion. Alternatively, in Relativistic Mechanics, \cite{lahiri04}
\begin{equation}
p^{\mu}p_{\mu}-m^2=0
\end{equation}
and
\begin{equation}
 p_{\mu} \xrightarrow{} i{\partial_{\mu}}
\end{equation}
in the position representation of momentum.
When this relation is applied to equation (1.2) we get equation (1.4) which describes the Hamiltonian of such a particle.
\begin{equation}
(\partial^{\mu}\partial_{\mu}+m^2)\Phi(x)=0
\end{equation}
%----------------------------------------------------------------------------------------

\section{spin-$ \frac{1}{2}$}
\subsection{Dirac Equation}
The Dirac Hamiltonian for a spin-$ \frac{1}{2} $ particle, commonly referred to as the "Dirac Equation" is \cite{lahiri04}
\begin{equation}
(i\gamma^{\mu}\partial_\mu-m)\psi(x)=0
\end{equation}
where
\begin{eqnarray}
\gamma^0=\begin{bmatrix}
1 & 0\\
0 & 1
\end{bmatrix}\\
\gamma^i=\begin{bmatrix}
0 &  \sigma_i\\
-\sigma_i & 0
\end{bmatrix}
\end{eqnarray}
The Dirac equation offers four interesting solutions instead of one that the Klein-Gordon field gives. These are, upon solving (1.5) \cite{griffiths08},
\begin{eqnarray}
\psi_a=\begin{bmatrix}
1\\
0\\
p_z/(p^0+m)\\
(p_x+ip_y)/(p^0+m)
\end{bmatrix}
\psi_b=\begin{bmatrix}
0\\
1\\
(p_x-ip_y)/(p^0+m)\\
-p_z/(p^0+m)\\
\end{bmatrix}\\
\psi_c=\begin{bmatrix}
p_z/(p^0+m)\\
(p_x+ip_y)/(p^0+m)\\
1\\
0
\end{bmatrix}
\psi_d=\begin{bmatrix}
(p_x-ip_y)/(p^0+m)\\
p_z/(p^0+m)\\
0\\
1
\end{bmatrix}
\end{eqnarray}
The first two equations describe the spin up ($\psi_a$) and spin down ($\psi_b$) particles respectively while the last two describe the spin up ($\psi_c$) and spin down ($\psi_d$) anti-particles.
\subsection{Dirac Field}
The Dirac Field for particle described by the Dirac equation is \cite{lahiri04}
\begin{equation}
{L} = \overset{-}{\psi}(i{\gamma^\mu}\partial_\mu-m)\psi
\end{equation}
The Euler-Lagrangian equations of (1.10) give rise to (1.5).
An interesting observation at this stage, which will be of huge significance in Chapter 3, is that the Dirac field is invariant under transformations of the kind
\begin{equation}
\psi \xrightarrow{} e^{-iq\theta}\psi
\end{equation}


This symmetry is an example of U(1) symmetry and the transformation elements are members of the U(1) symmetry group.
\subsection{Vector Fields}
It is enriching to explore Vector Fields, especially the Electromagnetic Field, at this stage. From, Classical Physics, Maxwell's equations are of the form \cite{griffiths13}
\begin{subequations}
Maxwell's equations in free space:
\begin{align}
{\nabla}.E=\rho,\\
{\nabla}.B=0\\
\frac{\partial B}{\partial t}&=-\nabla \times E,\\
\frac{\partial B}{\partial t}&=\nabla \times B - 4\pi j,
\end{align}
\end{subequations}
Potential can be defined for these fields such that
\begin{subequations}
\begin{align}
E={-\nabla}E+\frac{\partial A}{\partial t}\\
B=\nabla \times A
\end{align}
\end{subequations}
Defining a tensor $ F_{\mu\nu} $, called the field strength tensor while writing $A^{\mu}=(A^0,\overset{\xrightarrow{}}{A})$ such that, \cite{lahiri04}
\begin{equation}
F_{\mu\nu} = \partial_\mu{A_\nu}-\partial_\nu{A_\mu_}
\end{equation}
A Lagrangian can be written for the electromagnetic field as follows,
\begin{equation}
{L} = -\frac{1}{4}F_{{\mu}\nu}F^{{\mu}\nu}
\end{equation}
The Electomagnetic Field is known to be invariant under gauge transformations of the sort
\begin{equation}
{A^'}_\mu \xrightarrow{} {A_{\mu}}+{{\partial}_{\mu}}{\theta}
\end{equation} 
%----------------------------------------------------------------------------------------
